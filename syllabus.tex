\documentclass[letterpaper,oneside,onecolumn,11pt,article]{memoir}
\usepackage[T1]{fontenc}            % use T1 font encoding
\usepackage{textcomp}
\usepackage{courier}                % set courier as typewriter font
\usepackage{times}                  % set times as text font
\usepackage[scaled=0.92]{helvet}    % set Helvetica as the sans-serif font
\usepackage{mtpro2}

\usepackage{setspace}
\usepackage{amsmath}
\usepackage{graphicx,color}
\usepackage{wallpaper}
\usepackage{textcomp}
\usepackage{relsize,fancyvrb}
\usepackage{verbatim}
\usepackage{caption}
\usepackage{paralist}

\usepackage{boxedminipage}

\usepackage[bookmarks=true]{hyperref}
\hypersetup{
    unicode=false,          % non-Latin characters in Acrobat’s bookmarks
    pdftoolbar=true,        % show Acrobat’s toolbar?
    pdfmenubar=true,        % show Acrobat’s menu?
    pdffitwindow=true,      % page fit to window when opened
    pdftitle={Syllabus: Chemistry 350 / Fall 2013}, 
    pdfauthor={Dale J. Brugh},     % author
    pdfsubject={Physical Chemistry},   % subject of the document
    pdfnewwindow=true,      % links in new window
    pdfkeywords={classes, ch350f13}, % list of keywords
    colorlinks=true,       % false: boxed links; true: colored links
    linkcolor=black,          % color of internal links
    citecolor=green,        % color of links to bibliography
    filecolor=magenta,      % color of file links
    urlcolor=black           % color of external links
}


\definecolor{nicered}{rgb}{.647,.129,.149}
\definecolor{mutedgrey}{rgb}{0.4,0.4,0.4}
\definecolor{shadecolor}{cmyk}{0,0,0.25,0.07}
\definecolor{MyDarkBlue}{rgb}{0,0.08,0.45}
\definecolor{MarginRed}{rgb}{0.8,0.0,0.0}
\definecolor{MarginBlue}{rgb}{0.2,0.0,1.0}
\definecolor{MarginGrey}{rgb}{0.4,0.4,0.4}

%\renewcommand{\chapnumfont}{\bfseries\Huge\sffamily}
%\renewcommand{\chaptitlefont}{\bfseries\Large\sffamily}

\setsecheadstyle{\bfseries\Large\sffamily\raggedright}
\setsubsecheadstyle{\bfseries\large\sffamily\raggedright}
\setsubsubsecheadstyle{\bfseries\normalsize\sffamily\raggedright}
\renewcommand \thesection{\bfseries\arabic{section}}

\makeatletter 
\newcommand\addRevisionData{%
\begin{picture}(0,0)% 
    \put(-110,-5){%
        \tiny% 
        {%
        {Published \today \enspace \copyright~Dale J. Brugh 
        }
}% 
}%
\end{picture}%
}

\flushbottom
\setstocksize{11in}{8.5in}
%\setlength{\parskip}{5pt}
\settrims{0pt}{0pt}
%\settrimmedsize{11in}{210mm}{*}
%\setlength{\trimtop}{0pt}
%\setlength{\trimedge}{\stockwidth}
%\addtolength{\trimedge}{-\paperwidth}
\settypeblocksize{8.5in}{5.0in}{*}
\setulmargins{1.25in}{*}{*}
\setlrmargins{1.25in}{*}{*}
\setmarginnotes{5mm}{4.0cm}{\onelineskip}
\setheadfoot{\onelineskip}{4\onelineskip}
\setheaderspaces{*}{\onelineskip}{*}
\checkandfixthelayout
%\setlength \fboxsep{0.1in}
\setlength \headwidth{\textwidth+\marginparwidth+\marginparsep}
%
\makepagestyle{courseinformation}
\makerunningwidth{courseinformation}{\headwidth}

\makeheadrule{courseinformation}{\headwidth}{\normalrulethickness}
\makeheadposition{courseinformation}{flushright}{flushleft}{flushleft}{flushleft}
\makeoddhead{courseinformation}%
    {\sffamily Course Syllabus: Chemistry 350 / Fall 2013}{}{\sffamily\thepage}

    \makeevenfoot{courseinformation}{}{}{\addRevisionData}
    \makeoddfoot{courseinformation}{}{}{\addRevisionData}

\makepagestyle{courseinformationtitle}
\makerunningwidth{courseinformationtitle}{\headwidth}
\makeheadposition{courseinformationtitle}{flushright}{flushleft}{flushleft}{flushleft}
    \makeevenfoot{courseinformationtitle}{}{}{\addRevisionData}
    \makeoddfoot{courseinformationtitle}{}{}{\addRevisionData}

\pagestyle{courseinformation}

\captionsetup{labelsep=colon,aboveskip=0.25cm,justification=RaggedRight,singlelinecheck=false,labelfont={bf,sf}}

%===========================================================================
% Margin Figure Command
%===========================================================================
\newcommand{\marginfigures}[4]{
\marginpar{\centering
\includegraphics[width=#1]{#2}
\captionsetup{labelsep=newline,aboveskip=-0.5cm,justification=RaggedRight,singlelinecheck=false,labelfont={bf,sf}}
\captionsetup[figure]{position=bottom}
\captionof{figure}{#3}
\label{#4}}%
}%

%==========================================================================
% Margin Note Command
%==========================================================================

\newcommand{\marginnote}[2]
{%
\marginpar{\raggedright\vspace{#1}\begin{Spacing}{0.65}\sffamily{{\tiny$\blacktriangleright$~\scriptsize#2}}\end{Spacing}} %
}

%===========================================================================
% Set Up The Title
%===========================================================================
\setlength{\droptitle}{0.0in}
\backmatter
\pretitle{\noindent\huge\sffamily Course Syllabus \LARGE\par\noindent} 
\posttitle{\par\vskip 2.0em}
\preauthor{}
\postauthor{\par}
\predate{}
\postdate{\noindent\rule{\linewidth}{0.3pt}}

\title{Chemistry 350 / Fall 2013}
\date{}
\author{}

%============================================================================
% Begin Document
%============================================================================

\begin{document}
\setsecnumdepth{subsubsection}
\maketitle
%\setsecnumdepth{subsection}
\thispagestyle{courseinformationtitle}

\section{Instructor}
\begin{tabular}{rl|rl}
Name: & Dr. Dale J. Brugh & Office Location: & SCSC 262 \\
Email: & \href{mailto:djbrugh@owu.edu}{djbrugh@owu.edu} & Office Phone: & 740-368-3530 \\
%AIM: & roteric11 & Office Hours: & As Posted\\
\end{tabular}

\section{Meetings}
%\renewcommand{\arraystretch}{1.2}
\begin{tabular}{crcrl}
MWF & 10:00 am & to & 10:50 am & SCSC 167\\
Tu & 12:10 pm & to & 1:00 pm & SCSC 167 \\
\end{tabular} \\[0.09in]
Review meetings are scheduled as needed on either Sunday or Tuesday evening from 7:00 p.m.\ to 9:00 p.m. Review meetings are optional.

\section{Website}
The course website is located at \href{http://dephlo.net/pchem}{dephlo.net/pchem}. The website is an important extension of the syllabus and should be read carefully. 

\section{Prerequisites}
To \marginnote{-0.1in}{If you have not met these prerequisites, see me immediately.}take this course you must have passed two semesters of general chemistry, organic chemistry, calculus, and physics.

\section{Materials}
The following items are required for this course. Additional details can be found on the course website at \href{http://dephlo.net/pchem-materials}{dephlo.net/pchem-materials}.
\begin{enumerate}
\item \emph{Physical Chemistry} by Thomas Engel and Phillip Reid, Third Edition, Prentice Hall (2013).  ISBN-13: 9780321766205
\item \emph{Quanta, Matter, and Change} by Peter Atkins, Julio de Paula, and Ronald Friedman, First Edition, W.H. Freeman and Company (2009). ISBN-10: 07167-6117-3, ISBN-13: 978-0-7167-6117-4.

\item Wolfram's \emph{Mathematica}.

\item A working scientific calculator.
\end{enumerate}

\section{Time Requirement}

Each \marginnote{-0.1in}{Some students allow the assigned work in this class to fill all available time. This makes the course seem very time consuming. If you learn to focus properly, the work for this course can be completed in a reasonable time.} of the four weekly course meetings is 50 minutes in length, requiring a total of 3.33 hours per week. Review meetings will, on average, add another hour per week. For each meeting you can expect to spend at least 2 hours outside of class reading, studying, and working exercises. Problem sets will require five to ten hours during weeks they are assigned. If your study habits are not well-developed, more time may be required. This course will typically require work over weekends and University breaks.

\section{Content}

This course is an introduction to quantum mechanics, spectroscopy, and bonding applied to understanding chemical systems. The course topics are listed below. Topics are not necessarily covered in this order, and not all topics are covered with equal depth.
\begin{table}[h]
%\caption{\sffamily Topics covered in Chemistry 350.}
%\label{tab:topics}
\renewcommand{\arraystretch}{1}
\begin{tabular}{l|l} \toprule
Differential Equations & Atomic Spectroscopy \\
Beginnings of Quantum Mechanics & Computational Chemistry \\
Postulates of Quantum Mechanics & Molecular Spectroscopy \\
Schr\"{o}dinger's Wave Equation & Infrared Spectroscopy \\
Particle in a Box & Microwave Spectroscopy \\ 
Tunneling & Electronic Spectroscopy \\ 
Harmonic Oscillator & nmr Spectroscopy \\
Rigid Rotor & Variational Theory \\
Hydrogen Atom & Perturbation Theory \\
Multi-Electron Atoms &  Bonding in Diatomic Molecules \\
Atomic Term Symbols &  Bonding in Polyatomic Molecules \\
\bottomrule
\end{tabular}
\end{table}
\section{Goals}
Quantum \marginnote{-0.1in}{Quantum mechanics gives us the ability to predict chemistry without doing any wet laboratory work. What's not to like about that?} mechanics is the theoretical framework that organizes chemistry, starting with the periodic table. It allows us to understand the macroscopic behavior of matter in terms of the structure and interaction of individual atoms and molecules. It allows us to make sense of the interaction between matter and electromagnetic radiation. Its wide-ranging applicability makes it impossible to avoid quantum mechanics in any modern study of chemistry. It also happens to be a very cool achievement of the human mind. 

I want to share some of this with you. During this course I want to 
\begin{inparaenum}[\bfseries (a\upshape)]
\item provide you with the tools to understand how quantum mechanics applies to chemical systems;
\item provide you with the tools to understand quantum mechanics in the literature of any chemical subject;
\item show you the beauty and regularity of chemistry and Nature that emerge when you understand the most fundamental concepts that underlie everything you have ever experienced and everything you will ever experience;
\item provide you with the tools necessary to form a realistic mental image of the microscopic world of atoms and molecules;
\item further develop your ability to analyze and interpret experimental results in terms of the microscopic structure and dynamics of atoms and molecules; and
\item help you understand how we know what we know.
\end{inparaenum}

I also want you to move closer to being a professional in this course. Because of this, additional goals of this course are to
\begin{inparaenum}[\bfseries (a\upshape)]
\item improve your problem solving skills;
\item improve your ability to formulate descriptions of the physical and chemical world in terms of mathematical models; and
\item improve your ability to make clear and precise quantitative arguments.
\end{inparaenum}

\section{Learning Objectives}

Learning objectives are things you should be able to do at the end of the course. For each topic covered in this course, you are provided with a list of learning objectives called Be Able Tos, or BATS for short. A complete list of BATS for each course topic can be found at \href{http://dephlo.net/pchem-objectives}{dephlo.net/pchem-objectives}. These BATS are very detailed (granular), and it might be difficult to see the overall objectives of the course from them.

A higher level (less granular) list of learning objectives might be helpful. At the end of the course, you should be able to
\begin{inparaenum}[\bfseries (a\upshape)]
\item describe the difference between classical and quantum mechanics; 
\item describe the experiments that led to the development of quantum mechanics;
\item define wave-particle duality;
\item explain what quantization is and its origin;
\item solve the Schr\"odinger equation for a free particle and particle in a box;
\item use approximation techniques to solve the Schr\"odinger equation;
\item use molecular orbital theory to predict bonding in small molecules; and
\item predict and explain the outcome of electromagnetic radiation interacting with matter. 
\end{inparaenum}

\section{Weekly Routine}
The \marginnote{-0.1in}{The course website contains detailed schedules and lists of assignments.} Monday, Wednesday, and Friday meetings are dedicated to lecture. The Tuesday noon meeting is dedicated to a written micro exam or a lecture. An exercise assignment is due at the start of each MWF meeting. A problem set is assigned weekly and due each Saturday. An optional review session is offered Tuesday or Sunday evening. There are three exams, and they are on Tuesday outside of class time.

\section{Things I Grade}
You and I determine your progress in this course using scores derived from evaluating the quality and accuracy of your answers to questions posed in exercise assignments, problem sets, micro exams, exams, and a final exam. This section provides details for each of these evaluations

\subsection{Exercise Assignments}
Exercise assignments consist of no more than two exercises that are generally simple and straightforward to complete. Solutions should be worked out on paper and submitted at the start of the meeting for which they are assigned. Exercise assignments can be found on the course website. Solutions for exercise assignments are provided on the course website after I grade them. Each exercise assignment is worth 10 points. 

\subsection{Problem Sets}
Problem \marginnote{-0.1in}{The course website has detailed guidelines for preparing solutions for problems sets.} sets are challenging and more time-consuming than exercises. It is best to think of them as projects. Problem sets are typically assigned weekly and due every Saturday. Problem sets are posted on the course website. Solutions for problem sets are available in my office. Each problem set is worth 100 points.

\subsection{Micro Exams}
A micro exam is given about two weeks prior to each exam at the start of a Tuesday noon meeting. Exact dates are in the course schedule. Micro exams require the entire period to complete, and you are only given the class period to complete them. The topics covered on each micro exam are listed on the course website. Solutions to micro exams are posted on the course website after they are graded. Each micro exam is worth 50 points.

\subsection{Exams}
There are three exams during the semester. Each is administered during a three-hour block of time that you select on an announced Tuesday. All exams are cumulative. Solutions for exams are available in my office. See the course website for the exam schedule. Each exam is weighted equally. We do not meet as a class on Tuesday at noon on exam day.

\subsection{Final Exam}
The \marginnote{-0.1in}{Make plans to be on campus through the end of the final exam.} final exam is given on the day and at the time specified by the Registrar. There are no exceptions. The final exam is three hours in length, and it is cumulative over the entire semester. No solutions to the final exam are posted, but you may review the grading by making an appointment with me. You may not take possession of the exam.

\section{Things I Do Not Grade}
Each topic module is accompanied by a set of practice exercises. These exercises are recommended, but I never collect and grade them. It is essential that you work (or at least review) as many of the practice exercises as possible. Solutions for practice exercises from the course textbooks are available outside my office. Solutions for other practice exercises can be found on the course website.

\section{Course Score}
Your course score is a weighted average of the scores you earn on exercise assignments, problem sets, micro exams, and exams. These scores are weighted according to the percentages shown in Table~\ref{tab:weights}.
\begin{table}[h]
\caption{\sffamily Weight of items contributing to your course score.}
\label{tab:weights}
\begin{tabular}{r|l} \toprule
\textbf{Evaluation Item} & \textbf{Weight} \\ \hline
Exercise Assignments & 15\% \\
Problem Sets & 15\% \\
Micro Exams & 15\% \\
Exams & 30\% \\
Final Exam & 25\% \\
\bottomrule
\end{tabular}
\end{table}

\section{Letter Grade}
Letter grades are assigned at the end of the course according to the minimum course score requirements listed in Table~\ref{tab:lettergrades}. Course scores below $55\%$ are considered failing. Please see \href{http://dephlo.net/lettergrades}{dephlo.net/lettergrades} for more detail about how your course letter grade is determined. 

\begin{table}[h]
\caption{\sffamily Minimum course scores necessary for each letter grade.}
\label{tab:lettergrades}
\begin{tabular}{cl||cl} \toprule
\textbf{Minimum Score} & \textbf{Letter Grade} & \textbf{Minimum Score} & \textbf{Letter Grade} \\ \hline
97 & \hspace{0.3in}A$+$ & 72 & \hspace{0.3in}C$+$ \\
88 & \hspace{0.3in}A & 68 & \hspace{0.3in}C \\
85 & \hspace{0.3in}A$-$ & 65 & \hspace{0.3in}C$-$ \\
82 & \hspace{0.3in}B$+$ & 62 & \hspace{0.3in}D$+$ \\
78 & \hspace{0.3in}B & 58 & \hspace{0.3in}D \\
75 & \hspace{0.3in}B$-$ & 55 & \hspace{0.3in}D$-$ \\
\bottomrule
\end{tabular}
\end{table}

\section{Additional Information}

Please see the course website at \href{http://dephlo.net/pchem}{dephlo.net/pchem} for additional information such as suggestions for success, detailed course policies, course schedule, and solutions. 

\end{document}

% \section{Philosophy on Teaching and Learning}
% I want you to grow as a learner and person in this course. Guiding all my actions to help you achieve that growth is a philosophy about teaching and learning which can roughly be summarized in the following statements.
% \begin{itemize}
%     \item You \marginnote{-0.1in}{Implication: I do a lot of hard theory in class.} should be taught fundamental principles you cannot be expected to learn on your own so that you can build a foundation for understanding all the things in science and life that depend on these ideas.
%     \item You \marginnote{-0.1in}{Implication: I rarely work examples in class, and I sometimes expect you to learn topics on your own.} should not be taught the things you can be expected to learn on your own by application of fundamental principles.
%     \item You \marginnote{-0.1in}{Implication: Memorization is not as important as in some other classes.} should receive guidance in developing an
%     understanding of a subject in the context of what you already know,
%     but you should not be told what to think about a subject.
%     \item You \marginnote{-0.1in}{Implication: I \emph{really} pay attention to the quality of your mathematical and verbal arguments.} should be forced to defend your ideas based on clearly reasoned arguments rooted in the reality of the theories that humanity has documented up to the present day.
%     \item You \marginnote{-0.1in}{Implication: The course asks you to do a lot.} should be encouraged to perform at the limits of your potential.
%     \item You \marginnote{-0.1in}{Implication: I do not do your work for you.} should be encouraged to learn how to teach yourself, which is ultimately the only useful skill you will take away from OWU.
% \end{itemize}